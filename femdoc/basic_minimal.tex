
\importstr{basic}
{Example of a basic parameter input file ({\tt str} file)}

A basic version of an |str| file can be found in \ref{fig:basic}. In
fact, it is so basic, it really does not do anything. Here only the
compulsory parameters have been inserted. These are:

\begin{itemize}

\item An introductory section |$title| where on three lines the following
information is given:

\begin{enumerate}
\item A description of the run. This can be any text that fits on one line.
\item The name of the simulation. This name is used for all files that 
the simulation produces. These files differ from each other only by 
their extension.
\item The name of the basin. This is the basin file without the extension
|.bas|. The file must exist in the current directory.
\end{enumerate}

\item A section |$para| that contains all necessary parameters for the
simulation to be run. The only compulsory parameters are the ones that
specify the start of the simulation |itanf|, its end |itend|, its 
time step |idt| and a reference date (|date = yyyymmdd|). This is the
reference for all the time parameters used in the |str| file and in all 
the files provided as input to the simulation, as well as the output files.

\end{itemize}

All the time parameters can be specified in seconds from the reference
date or using a date label |'yyyy-mm-dd::HH:MM:SS'|. The parameters that specify
time steps can be prescribed both in seconds or using the following labels: 
|'Ns'|, |'Nm'|, |'Nh'|, |'Nd'|. 
Where |N| is a number, |s| means seconds, |m| minutes, |h| hours and |d| days.

In order to be more helpful, some more information must be added to the
|str| file. As an example let's have a look on \figref{example}. Here
we have added two parameters that deal with the type of friction
to be used. |ireib| specifies the bottom friction formulation, here
through a simple quadratic bulk formula. (For the exact meaning of the
parameters, please refer to the appendix where all parameters
are listed.) The parameter |czdef| specifies the value to use for the
bottom drag coefficient.

