
%------------------------------------------------------------------------
%
%    Copyright (C) 1985-2018  Georg Umgiesser
%
%    This file is part of SHYFEM.
%
%    SHYFEM is free software: you can redistribute it and/or modify
%    it under the terms of the GNU General Public License as published by
%    the Free Software Foundation, either version 3 of the License, or
%    (at your option) any later version.
%
%    SHYFEM is distributed in the hope that it will be useful,
%    but WITHOUT ANY WARRANTY; without even the implied warranty of
%    MERCHANTABILITY or FITNESS FOR A PARTICULAR PURPOSE. See the
%    GNU General Public License for more details.
%
%    You should have received a copy of the GNU General Public License
%    along with SHYFEM. Please see the file COPYING in the main directory.
%    If not, see <http://www.gnu.org/licenses/>.
%
%    Contributions to this file can be found below in the revision log.
%
%------------------------------------------------------------------------

This section explains typical usage of the model. It will show how
the model can be run doing basic 2D hydrodynamic simulations, simulate
a passive tracer, compute T/S, use the Coriolis force and apply wind
forcing. More advanced usages of the model, like 3D simulations and the
use of the turbulence module will be presented later. This section is
conceived as a simple HOWTO document. For the exact meaning and usage
of the single parameters, please see the section on input parameters.

To run a simulation, two things are needed. The first is the description
of the basin and the numerical grid, which must be prepared beforehand and
then must be compiled in a form that the model can use. How this is been
done has already been described in the chapter dealing with preprocessing.

The second thing that is needed is a description of the simulation and the
forcings that have to be applied. This is done through a parameter input
file. Here we call it |str| file, because historically these files always
ended with an extension of |.str|. However, any extension can be used.

