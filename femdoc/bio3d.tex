%\documentclass{report}
%\begin{document}
                                                                              
                                                                                                    
                                                                                                    
                                                                                                    
                                                                                                    
                                                                                                    
                                                                                                    
                                                                                                    
                                                                                                    
                                                                                                    
\newcommand{\tab}{\hspace{5mm}}

\newcommand{\Tone}{\ref{MassBalance}}
\newcommand{\Ttwoa}{\ref{FuncDesc}}
\newcommand{\Ttwob}{\ref{Paras}}
\newcommand{\Ttwoc}{\ref{Vars}}

\newcommand{\STone}{\ref{SMassBalance}}
\newcommand{\STtwoa}{\ref{SFuncDesc}}
\newcommand{\STtwob}{\ref{SParas}}
\newcommand{\STtwoc}{\ref{SVars}}




by Donata Melaku Canu, Georg Umgiesser, Cosimo Solidoro

\vspace{1cm}

The coupling between EUTRO and FEM constitute a structure which 
is meant to be a generic water quality for full eutrophication 
dynamics.
The Water Quality model is described fully in
Umgiesser et al. (2003).



\section{General Description}



The water quality model has been derived from the EUTRO module 
of WASP (released by the U.S. Environmental Protection Agency 
(EPA) (Ambrose et al., 1993) and modified. It simulates the evolution 
of nine state variables in the water column and sediment bed, 
including dissolved oxygen (DO), carbonaceous biochemical oxygen 
demand (CBOD), phytoplankton carbon and chlorophyll a (PHY), 
ammonia (NH3), nitrate (NOX), organic nitrogen (ON), organic 
phosphorus (OP), orthophosphate (OPO4) and zooplankton (ZOO). 
The interacting nine state variables can be considered as four 
interacting systems: the carbon cycle, the phosphorous cycle, 
the nitrogen cycle and the dissolved oxygen balance (Fig. ??). 
Different levels of complexity can be selected by switching the 
eight variables on and off, in order to address the specific 
topics.

The evolution of phytoplankton concentration (Reaction 1, 
Table \Tone)
is described by the anabolic and the catabolic terms, plus 
a grazing term related to zooplankton concentration (Reaction 
10, 11 and 12, Table \Ttwoa), which however is treated as a constant
in the original version. 
The anabolic term (Reaction 10, Table \Ttwoa) is related to light 
intensity, temperature and concentration of nutrients in water, 
while the catabolic term (Reaction 11, Table \Ttwoa) depends on temperature.

Phytoplankton growth is described by combining a maximum growth 
rate under optimal conditions, and a number of dimensionless 
factors, each ranging from 0 to 1, and each one referring to 
a specific environmental factor (nutrient, light availability), 
which reduces the phytoplanktonic growth insofar as environmental 
conditions are at sub-optimal levels. Phytoplankton stochiometry 
is fixed at the user-specified ratio, so that no luxury uptake 
mechanisms are considered, and the uptake of nutrients is directly 
linked to the phytoplankton growth, and described by the same 
one-step kinetic law. More specifically, the influence of inorganic 
phosphorous and nitrogen availability on phytoplankton growth/nutrients 
uptake is simulated by means of Michealis-Menten-Monod kinetics 
(Reactions 42 and 43, Table \Ttwoa). Phytoplankton uptakes nitrogen 
both in the forms of ammonia and nitrate, but ammonia is assimilated 
preferentially, as indicated in the ammonia preference relation 
(Reaction 38, Table \Ttwoa). The influence of temperature is given 
by an exponential relation (Reaction 13, Table \Ttwoa), while the 
functional forms for the limitation due to sub-optimal light 
condition can be chosen between three alternative options, namely 
the formulation proposed by Di Toro et al. (1971) and the one 
proposed by Smith (1980) (Di Toro and Smith subroutines, Reaction 
44, Table \Ttwoa) and the Steele formulation (Steele, 1962) that 
can use hourly light input values. The 
choice between different available functional forms (Ditoro, 
Smith, and Steele) is made by setting the index |LGHTSW| equal 
to 1, 2 or 3. The new version is therefore able to simulate diurnal 
variations depending on light intensity, such as night anoxia 
due to phytoplankton respiration during nighttime.

Finally, the two frequently used models for combining maximum 
growth and limiting factors, the multiplicative and the minimum 
(or Liebig's) model, are both implemented, and the user can choose 
which one to adopt (Reaction 41, Table \Ttwoa).

Nitrogen and phosphorous are then returned to the organic compartment 
(ON, OP) via phytoplankton and zooplankton respiration and death. 
After mineralization, the organic form is again converted into
the dissolved inorganic form available for phytoplankton growth. 

The DO mass balance is influenced by almost all of the processes 
going on in the system. The reaeration process acts to restore 
the thermodynamic equilibrium level, the saturation value, while 
respirations activities and mineralization of particulated and 
dissolved organic matter consume DO and, of course, photosynthetic 
activity produces it. Other terms included in the DO mass balance 
are the ones referring to redox reactions such as nitrification 
and denitrification. The reaeration rate is computed from the 
model in agreement with either the flow-induced rate or the wind-induced 
rate, whichever is larger. The wind-induced reaeration rate is 
determined as a function of wind speed, water and air temperature, 
in agreement with O'Connor (1983), while the flow-induced reaeration 
is based on the Covar method (Covar, 1976), i.e., it is calculated 
as a function of current velocity, depth and temperature.

The dynamic of a generic herbivorous zooplankton compartment 
(ZOO), meant to be representative of the pool of all the herbivorous 
zooplankton species, is followed and accordingly the subroutines 
relative to phytoplankton, organic matter, nutrients, and dissolved 
oxygen, which were influenced by such a modification. 

The grazing has been described by means of a type II functional 
relationship, as it is usually done for aquatic ecosystems. However, 
the possibility to select a type III relationship, as well as 
to maintain the original parameterisation of constant zooplankton, 
has been included.  

The zooplankton assimilates the ingested phytoplankton with an 
efficiency EFF, and the fraction not assimilated, ecologically 
representative of faecal pellets and sloppy feeding, is transferred 
to the organic matter compartments (dotted lines Fig. ??). Finally, 
zooplankton mortality is described by a first order kinetics. 
The code has been written by adopting the standard WASP nomenclature 
system, and the choice between the different available functional 
forms is performed by setting the index |IGRAZ|. A choice of 0 
(the default value) corresponds to the original EUTRO version, 
giving the user the ability to chose easily between the extended 
version or revert to the original one.



%------------------------------------------------------------------------
%
%    Copyright (C) 1985-2020  Georg Umgiesser
%
%    This file is part of SHYFEM.
%
%    SHYFEM is free software: you can redistribute it and/or modify
%    it under the terms of the GNU General Public License as published by
%    the Free Software Foundation, either version 3 of the License, or
%    (at your option) any later version.
%
%    SHYFEM is distributed in the hope that it will be useful,
%    but WITHOUT ANY WARRANTY; without even the implied warranty of
%    MERCHANTABILITY or FITNESS FOR A PARTICULAR PURPOSE. See the
%    GNU General Public License for more details.
%
%    You should have received a copy of the GNU General Public License
%    along with SHYFEM. Please see the file COPYING in the main directory.
%    If not, see <http://www.gnu.org/licenses/>.
%
%    Contributions to this file can be found below in the revision log.
%
%------------------------------------------------------------------------

%\documentclass{report}
%\usepackage{a4}
%\usepackage{shortvrb}
%\begin{document}



\newcommand{\Otwo}{O${}_{2}$}
\newcommand{\Degree}{${}^{o}$}
\newcommand{\power}[1]{${}^{#1}$}

\newcommand{\opn}{ {} }


\newcommand{\VSPGB}{\vspace{0.2cm}}
\newcommand{\HSP}{\hspace*{1.5cm}}
\newcommand{\HHSP}{\hspace*{0.5cm}}
\newcommand{\GBox}[2]{\parbox{#1 cm}{\VSPGB#2\VSPGB}}
\newcommand{\GDBox}[1]{\GBox{5}{#1}}
\newcommand{\GEBox}[1]{\GBox{7}{#1}}
\newcommand{\GFBox}[1]{\GBox{7}{#1}}



\begin{table}\centering
\begin{tabular}{lll}
\hline


& & \\
$\frac{\partial S}{\partial t} =Q(S)$
& &
General Reactor Equation
\\
& & \\

$Q(PHY) = GPP - DPP - GRZ$
& 1 & 
Phytoplankton PHY [mg C/L]
\\

$Q(ZOO) = GZ - DZ$
& 2 &
Zooplankton ZOO [mg C/L]
\\

$Q(NH3) = N_{alg1} + ON1 - N_{alg2} - N1$
& 3 &
Ammonia NH3 [mg N/L]
\\

$Q(NOX) = N1 - NO_{alg} - NIT1$
& 4 &
Nitrate NOX [mg N/L]
\\

$Q(ON) = ON_{alg} - ON1$
& 5 &
Organic Nitrogen ON [mg N/L]
\\

$Q(OPO4) = OP_{alg1} + OP1 - OP_{alg2}$ 
& 6 &
\GBox{5}{
Inorganic Phosphorous OPO4 \\
\HHSP [mg P/L]
}
\\

$Q(OP) = OP_{alg3} - OP1$
& 7 &
Organic Phosphorous OP [mg P/L]
\\

$Q(CBOD) = C1 - OX - NIT2$
& 8 &
\GBox{5}{
Carbonaceous Biological Oxygen \\
\HHSP Demand CBOD [mg \Otwo/L]
}
\\

\GBox{6}{
$Q(DO) = DO1 + DO2 + DO3 $\\
\HSP $\opn - DO4 - N2 - OX - SOD$
}
& 9 &
Dissolved Oxygen DO [mg \Otwo/L]
\\


\hline
\end{tabular}
\caption{Mass balances}
\label{MassBalance}
\end{table}






\begin{table}\centering
\begin{tabular}{lll}
\hline


$GPP=GP1*PHY $
& 10 &
phytoplankton growth
\\

$DPP=DP1*PHY $
& 11 &
phytoplankton death 
\\

$GRZ=KGRZ*\frac{PHY}{PHY+KPZ}*ZOO$
& 12 &
grazing rate coefficient
\\

$GP1=L_{nut} *L_{light} *K1C*K1T^{(T-T_{0} )} $
& 13 &
\GDBox{
phytoplankton growth rate with nutrient and light limitation
}
\\

$DP1=RES+K1D$ 
& 14 &
\GDBox{
phytoplankton respiration and death rate
}
\\

$GZ=EFF*GRZ$ 
& 15 &
zooplankton growth rate
\\

$DZ=KDZ*ZOO$ 
& 16 &
zooplankton death rate
\\

$Z_{ineff} = (1-EFF)*GRZ $
& 17 &
grazing inefficiency on phytoplankton
\\

$Z_{sink} = Z_{ineff}+DZ $
& 18 &
sink of zooplankton
\\

$N_{alg1}= NC*DPP*(1-FON) $
& 19 &
source of ammonia from algal 
death
\\

$N_{alg2}=PN*NC*GPP $
& 20 &
sink of ammonia for algal growth
\\

$NO_{alg}= (1. - PN)*NC*GPP $
& 21 &
sink of nitrate for algal growth
\\

$ON_{alg}= NC*(DPP*FON+Z_{sink}) $
& 22 &
\GDBox{
source of organic nitrogen from phytoplankton and zooplankton death
}
\\

\GBox{5}{
$N1=KC_{nit} *KT_{nit}^{(T-T_{0})} *NH3$
\HSP $\opn *\frac{DO}{K_{nit}+DO}$
}
& 23 &
nitrification
\\

\GBox{5}{
$NIT1=KC_{denit} KT_{denit}^{(T-T_{0})}$
\HSP $\opn *NOX*\frac{K_{denit}}{K_{denit}+DO}$
}
& 24 &
denitrification
\\

$ON1=KNC_{\min } *KNT_{\min }^{(T-T_{0})} *ON$
& 25 &
mineralization of ON
\\

$OP1=KPC_{\min } *KPT_{\min } ^{(T-T_{0})} *OP$
& 26 &
mineralization of OP
\\

$OP_{alg1}=PC*DPP*(1. - FOP) $
& 27 &
\GDBox{
source of inorganic phosphorous from algal death
}
\\

$OP_{alg2}=PC*GPP $
& 28 &
\GDBox{
sink of inorganic phosphorous for algal growth
}
\\

$OP_{alg3}=PC*(DPP*FOP+Z_{sink}) $
& 29 &
\GDBox{
source of organic phosphorous from phytoplankton and zooplankton death
}
\\

\GBox{5}{
$OX=KDC*KDT^{(T-T_{0} )} $
\HSP $\opn *CBOD*\frac{DO}{KBOD+DO} $
}
& 30 &
oxidation of CBOD
\\

\hline
\end{tabular}
\caption{Functional Expression Description}
\label{FuncDesc}
\end{table}

\begin{table}\centering
\begin{tabular}{lll}
\hline

$C1=OC*(K1D*PHY+Z_{sink}) $
& 31 &
\GDBox{
source of CBOD from phytoplankton and zooplankton death
}
\\

$NIT2=\left( \frac{5}{4} *\frac{32}{14} *NIT1\right) $
& 32 &
sink of CBOD due to denitrification
\\

$DO1=KA*(O_{sat} - DO) $
& 33 &
reareation term
\\

$DO2=PN*GP1*PHY*OC $
& 34 &
\GDBox{
dissolved oxygen produced by phytoplankton using NH3
}
\\

\GBox{5}{
$DO3=(1-PN)*GP1*PHY$
\HSP $\opn * 32*\left( \frac{1}{12} +1.5*\frac{NC}{14} \right) $
}
& 35 &
growth of phytoplankton using NOX
\\

$DO4 = OC*RES*PHY$ 
& 36 &
respiration term
\\

$N2=\left( \frac{64}{14} *N1\right) $
& 37 &
oxygen consumption due to nitrification
\\

\GBox{6}{
$PN=\frac{NH3*NOX}{(KN+NH3)*(KN+NOX)}$ 
\HSP $\opn +\frac{NH3*KN}{(NH3+NOX)*(KN+NOX)} $
}
& 38 &
ammonia preference
\\

$RES=K1RC*K1RT^{(T-T_{0})} $
& 39 &
algal respiration
\\

$SOD=\frac{SOD1}{H} *SODT^{(T-T_{0})} $
& 40 &
sediment oxygen demand
\\

$L_{nut}= min(X1,X2) \, , \,  mult(X1,X2) $
& 41 &
\GDBox{
minimum or multiplicative nutrient limitation for phytoplankton growth
}
\\

$X1=\frac{NH3+NOX}{KN+NH3+NOX} $
& 42 &
\GDBox{
nitrogen limitation for phytoplankton growth
}
\\

$X2=\frac{OPO4}{\frac{KP}{FOPO4} +OPO4} $
& 43 &
\GDBox{
phosphorous limitation for phytoplankton growth
}
\\

$L_{light} =\frac{I_{0} }{I_{s} } *e^{-(KE*H)} *e^{(1-\frac{I_{0} }{I_{s}
} *e^{(-KE*H)} )} $
& 44 &
\GDBox{
light limitation for phytoplankton growth
}
\\

$KA=F(Wind,Vel,T,T_{air},H) $
& 45 &
re-areation coefficient 
\\


\hline
\end{tabular}
\addtocounter{table}{-1}
\caption{(continued) Functional Expression Description}
\label{FuncDesc1}
\end{table}













\begin{table}\centering
\begin{tabular}{ll}
\hline


$K1D=0.12$ day\power{-1}  
&
phytoplankton death rate constant
\\

$KGRZ=1.2$ day\power{-1}  
&
grazing rate constant
\\

$KPZ=0.5$ mg C/L 
&
\GFBox{
half saturation constant for phytoplankton in grazing
}
\\

$KDZ=0.168$ day\power{-1} 
&
zooplankton death rate
\\

$K1C=2.88$ day\power{-1} 
&
phytoplankton growth rate constant
\\

$K1T=1.068$ 
&
\GFBox{
phytoplankton growth rate temperature constant
}
\\

$KN=0.05$ mg N/L 
&
\GFBox{
nitrogen half saturation constant for phytoplankton growth
}
\\

$KP=0.01$ mg P/L 
&
\GFBox{
phosphorous half saturation constant for phytoplankton growth
}
\\

$KC_{nit}=0.05$ day\power{-1} 
&
nitrification rate constant
\\

$KT_{nit}=1.08$ 
&
nitrification rate temperature constant
\\

$K_{nit}=2.0$ mg \Otwo/L 
&
half saturation constant for nitrification
\\

$KC_{denit}=0.09$ day\power{-1} 
&
denitrification rate constant
\\

$KT_{denit}=1.045$ 
&
denitrification rate temperature constant
\\

$K_{denit}=0.1$ mg \Otwo/L 
&
half saturation constant for denitrification
\\

$KNC_{min}=0.075$ day\power{-1} 
&
mineralization of dissolved ON rate constant
\\

$KNT_{min}=1.08$ 
&
\GFBox{
mineralization of dissolved ON rate temperature constant
}
\\

$KDC=0.18$ day\power{-1}
&
oxidation of CBOD rate constant
\\

$KDT= 1.047$ 
&
oxidation of CBOD rate temperature constant
\\

$NC=0.115$ mg N/mg C 
&
N/C ratio
\\

$PC=0.025$ mg P/mg 
&
C P/C ratio
\\

$OC=32/12$ mg \Otwo/mg C 
&
O/C ratio
\\

$EFF=0.5$ 
&
grazing efficiency
\\

$FON=0.5$ 
&
fraction of ON from algal death
\\

$FOP=0.5$ 
&
fraction of OP from algal death
\\

$FOPO4=0.9$ 
&
fraction of dissolved inorganic phosphorous
\\

$KPC_{min}=0.0004$ day\power{-1}  
&
mineralization of dissolved OP rate constant
\\

$KPT_{min}=1.08$ 
&
\GFBox{
mineralization of dissolved OP rate temperature constant
}
\\

$KBOD=0.5$ mg \Otwo/L
&
CBOD half saturation constant for oxidation
\\

$K1RC=0.096$ day\power{-1}  
&
algal respiration rate constant
\\

$K1RT=1.068$ 
&
algal respiration rate temperature constant
\\

$I_{s}=1200000$ lux/day
&
\GFBox{
optimal value of light intensity for phytoplankton growth
}
\\

$KE=1.0$ m\power{-1}
&
light extinction coefficient
\\

\GBox{5}{
$SOD1=2.0$ mg \Otwo/L \\
\HSP day\power{-1} m 
}
&
sediment oxygen demand rate constant
\\

$SODT=1.08$ 
&
sediment oxygen demand temperature constant
\\

$T_{0}=20$ \Degree C 
&
optimal temperature value
\\


\hline
\end{tabular}
\caption{Parameters}
\label{Paras}
\end{table}





\begin{table}\centering
\begin{tabular}{lll}
\hline


$T$ 
& [\Degree C] &
water temperature
\\

$T_{air}$ 
& [\Degree C] &
air temperature
\\

$O_{sat}$  
& [mg/L] &
DO concentration value at saturation
\\

$I_{0}$ 
& [lux/day] &
incident light intensity at the surface
\\

$H$ 
& [m] &
depth
\\

$Vol$ 
& [m\power{3}] &
volume
\\

$Vel$ 
& [m/sec] &
current speed
\\

$Wind$ 
& [m/sec] &
wind speed
\\


\hline
\end{tabular}
\caption{Variables}
\label{Vars}
\end{table}




%\end{document}



\section{The coupling}


Mathematical models usually describe the coupling between ecological 
and physical process by suitable implementation of an advection/diffusion 
equation for a generic tracer, reads 

\begin{equation} \label{AdvDif}
\frac{\partial \,\Theta _{i} }{\partial \,t} \,\,+U\cdot \nabla \,\Theta
_{i} -\,w^{s}_{i} \,\frac{\partial \,\Theta _{i} }{\partial \,z} \,=\,\,K_{h}
\,\nabla _{H}^{2} \Theta _{i} \,+\,\frac{\partial \,}{\partial \,z}
\,\left[ K_{v} \,\frac{\partial \,\Theta _{i} }{\partial \,z} \right]
\,+\,F\,\left( \Theta \,,\,T,\,I\,,\,\,..\right)
\end{equation}

where $U$ is the (average components of the) velocity, 
the $\Theta_{i}$ are the tracers which compose the entire 
vector of the biological state variable $\Theta$ and 
$F$ is a source term. $T$ and $I$ indicate, respectively, 
water temperature and Irradiance level, while $w^{s}_{i}$ represent 
the downward flux rates (sinking velocity) for the tracer 
$\Theta_{i}$, 
and $K_{h}$ and $K_{v}$ are the eddy coefficients for 
horizontal and vertical turbulent diffusion.

The term $F$ includes the contributions of the biological/biogeochemical 
activities, and the whole biological state vector $\Theta$ 
is explicitly considered in the last term of equation \ref{AdvDif}, without 
a spatial operator. As far as the biologically induced variations 
are regarded, the fate of each tracer in every location $x,y,z$ 
is tightly coupled to other tracers in the same location, but 
is not directly influenced by processes going on elsewhere. 

Therefore, in this approximation the global temporal variation 
of any tracer (state variable, conservative or not) can be split 
into the sum of two independent contributions: 

\begin{equation}
\frac{\partial \,\Theta _{i} }{\partial \,t} \,=\,\,\left. \frac{\partial
\,\Theta _{i} }{\partial \,t} \right\arrowvert_{phys} +\left. \frac{\partial
\,\Theta _{i} }{\partial \,t} \right\arrowvert_{biol}
\end{equation}

and it might be convenient, in writing a computer code, to devote 
independent modules to computation of each of them. Indeed, most 
of the modern water quality programs do have, at least conceptually, 
a modular structure. In this way the same code can be used for 
simulating different situations: by switching off the module 
referring to the reactor term the transport of a purely passive 
tracer is reproduced, while a 0D, close and uniformly stirred 
biological system is simulated if the module referring to the 
physical term is not included. Finally, the inclusion of both 
modules gives the evolution of tracers subjected to both physical 
and biogeochemical transformation, in a representation that, 
depending upon the parameterisation of the physical module, can 
be 1, 2 or 3 dimensional.

The whole water quality module is contained in a file
|weutro.f| and the call to EUTRO is made through a subroutine call
that is done from the main program through an appropriate 
interface. There is a clean division between the hydrodynamic 
motor, parameters used by the model and the resolution of the 
differential equations and the ecological model as evidenced 
by the overall structure of the modules. 

It is the responsibility of the main module to implement the 
time loop administration, the advective and diffusive transport 
of the state variables, both in the horizontal and vertical direction 
and the application of the boundary conditions. 

The typical use of the new EUTRO module is as follows: the main 
program first sets all parameters needed in EUTRO through the 
call to |EUTRO_INI|. These parameters are the kinetic constants 
of the reactions that are described in EUTRO and are considered 
constant for a site. They have to be set therefore only once 
at the beginning of the simulation. Once set, these parameters 
are available to the EUTRO module as global parameters.

For every box in the discretized domain (horizontal and vertical), 
and for every time step, the main program calls the subroutine 
|EUTRO0D|. Inside |EUTRO0D| the differential equations that describe 
the bio-chemical reactions are solved with a simple Euler scheme.

The values passed into |EUTRO0D| can be roughly divided into 4 
groups. The first group is made out of the aforementioned constants 
that represent the kinetic constants and other parameters that 
do not vary in time and space. The second group represents the 
state variables that are actually modified by |EUTRO0D| through 
the bio-chemical reactions. These variables are transported and 
diffused by the main routine and are just passed into |EUTRO0D| 
for the description of the processes. After the call no memory 
remains in |EUTRO0D| of these state variables. They must therefore 
be stored away by the main routine to be used in the next time 
step again. The third and fourth groups of values have to do 
with the forcing terms. They have been divided in order to account 
for the different nature of the forcing terms. The third group 
consists of the hydrodynamic forcing terms that are directly 
computed by the hydrodynamic model and parameters related to 
the box discretization. They consist of water temperature, salinity, 
current velocity, and depth and type of the box. Here the type 
identifies the position of the box (surface, water column, sediment), 
which is needed for some of the forcings to be applied. These 
variables are passed directly into EUTRO through a parameter 
list. The last group contains other forcing terms that are not 
directly related to the hydrodynamic model. These consist of 
the meteorological forcings (wind speed, air temperature, ice 
cover), light climate (surface light intensity, day length) and 
sediment fluxes. These parameters are set through a number of 
commodity functions that are called by the main routine. The 
reason why the last two parameter groups are handled differently 
from each other has also to do with the fact that the third group 
is highly variable in time and space. Variables like current 
velocity change with every time step and are normally different 
from element to element. The fourth group is very often only 
slowly variable in time (light, wind) and can very often be set 
constant in space. Therefore these values can be set at larger 
intervals, and do not have to be changed when looping over all 
the elements in the domain.

The overall flow of information during one time step is the following: 
First the hydrodynamic model resolves the momentum and continuity 
equation to update the current velocities and water levels. After 
that the physical (temperature and salinity) and bio-chemical 
scalars are advected and diffused. Once this advection step has 
been handled the new loadings and forcing terms are set-up and 
then |EUTRO0D| is called for the bio-chemical reactions. 


Note that the operator splitting technique, which decouples the 
advective and diffusive transport from the source term, allows 
for different time steps of the two processes for a more efficient 
use of the computer resources. 


Default values of the water quality parameters are already set 
in the code. Owner specific parameters for the water quality 
model should be written in the subroutine |param_user|.





\section{Light limitation}

\input steele.tex



\section{Initialization}



This section describes
the interpolation of data for the initialisation of the model.

To create a file with initial conditions the program
|laplap| can be used. The program is called as
|laplap < namefile.dat|.
This makes a laplacian interpolation of specified data contained in 
the |namefile.dat|.
This data file should have the first line empty and shold contain 
two colums containing, respectively, node number and data values for
the node.

It generates two files, |laplap.nos| and |laplap.dat|. The first 
one can be used to check if the procedure has been conducted 
well, creating a map with the plotting procedure (see Postprocessing 
section). The |.dat| file name should be given in the section
|$name| 
of the |.str| file to initialize the model.

You can create initialisation files for temperature, salinity, 
wind field and biological variables.
If you want to initialise the biological model with biological 
data you should create a single data file merging the 9 data 
files (one for each variable) using the |inputmerge.f| routine.




\newcommand{\listroutine}[1]{\paragraph{\texttt{#1}}}
\newcommand{\lt}{$<$}
\newcommand{\gt}{$>$}



\section{Post processing}

This section shows how to generate derivate variables.

The post processing routines elaborate the water quality outputs 
to generate derivate variables. They allow to generate variables 
such as averages, (both, in time and in space), sum differences, 
and water quality variables such us Vismara, TRIX and BOD5.

The routines and their usage are the following:

\listroutine{nosmaniav.f}
It generates a file containing, for each node of the spatial 
domani, average, minimum and maximum values of the specified 
variable of the whole simulation.

\listroutine{nosmaniqual.f}
It generates a water quality file from the elaboration of the 
state variable.
It computes, for each node, and at each time step a water quality 
index that can be chosen between two suggested indexes: Vismara 
QualityV and TRIX a well known quality index applied to the 
water quality definition at coastal seas and estuaries.

These indices can be computed using the definitions
in Table \ref{ClassWQI} and
the following equations:

\begin{center}
\begin{verb}
QualityV = class(O2sat) + class(BOD5) + class(NH3)
\\
auxt1 = (phyto/30.)*1000\\
auxt2 = (nh3+nox)*1000\\
auxt3 = (opo4+op)*1000\\
TRIX = log10(auxt1*o2satp*auxt2*auxt3+1.5)/1.2
\end{verb}
\end{center}


\begin{table}\centering
\begin{tabular}{lccccc}
\hline

Class(Var) & 1 & 2 & 3 & 4 & 5\\
O2sat & 90-110 & 70-90 or 110-120 & 50-70 or 120-130 & 30-50 & \lt30 or \gt130\\
BOD5 & \lt3 & 3-6 & 6-9 & 9-15 & \gt15\\
NH3 & \lt0.4 & 0.4-1 & 1-2 & 2-5 & \lt5\\

\hline
\end{tabular}
\caption{Classification of Water Quality Indices}
\label{ClassWQI}
\end{table}




\listroutine{nosmanintot.f}
generates a file of total inorganic Nitrogen as sum of NH3 and 
NOx

\listroutine{nosdif.f}
computer for the chosen state variable, the difference between 
the values at two times step. 


\listroutine{nosdiff.f}
computer the difference between the variable outputs of two simulations 



\listroutine{nosmanibod5.f}
computes the BOD5 values from the CBOD outputs as:

\begin{center}
\begin{verb}
bod5 = cbod*(1. - exp( -5. * par1 ))
\\
        + (64./14.) * nh3 * (1. - exp( -5. * par2 ))
\end{verb}
\end{center}

To run one of the postprocessing routine write the name of the 
routine and enter.




\section{The Sediment Module}

The sediment buffer action on the biogeochemical cycles could 
be very important, especially in the shallow water basins and 
during the storm surge events.

The routine |wsedim| (introduced in April 2004) aims to address 
the resuspention/sinking dynamics of nitrogen and phosphorous.
This routine can be switched on and off as needed by the user, 
setting the |bsedim| parameter |true| or |false| in the |bio3d| 
routine. It is called after the the eutrophication subroutine.

It allows to follow the dynamics of two additional variables, 
OPsed and ONsed that simulate the evolution of Nitrogen and Phosphorous 
detritus in sediment that are not subjected to advection-diffusion 
processes.

These two variables interact with the Nitrogen and Phosphorous 
cycle as decribed by the equations in Table \STone. When 
the |wsedim| subroutine is switched on, OP, ON, NH3 and OPO4 are 
updated at each time step in agreement with those equations.

The resuspension is a linear function of the water velocity calculated 
by the hydrodynamic model at each box, as written in Table \STtwoa. 
The amount of the sinking nutrients depends on specific prossess 
parameters, as given in Table \STtwoa, and on the depth of the underlying 
column.


%\documentclass{report}
%\usepackage{a4}
%\usepackage{shortvrb}
%\begin{document}
%
%\newcommand{\Degree}{${}^{o}$}
%\newcommand{\power}[1]{${}^{#1}$}
%
%\newcommand{\VSPGB}{\vspace{0.2cm}}
%\newcommand{\HSP}{\hspace*{1.5cm}}
%\newcommand{\HHSP}{\hspace*{0.5cm}}
%\newcommand{\GBox}[2]{\parbox{#1 cm}{\VSPGB #2 \VSPGB}}
%\newcommand{\GDBox}[1]{\GBox{5}{#1}}
%\newcommand{\GEBox}[1]{\GBox{7}{#1}}
%\newcommand{\GFBox}[1]{\GBox{7}{#1}}
\newcommand{\GHBox}[1]{\GBox{5.2}{#1}}
%
%\newcommand{\opn}[1]{\, #1 \,}
%\newcommand{\opn}{ {} }
%\newcommand{\opn}[1]{#1}



\begin{table}\centering
\begin{tabular}{lll}
\hline


& & \\
$\frac{\partial S}{\partial t} =Q(S)_{sed}$
& &
General Reactor Equation
\\
& & \\

$Q(NH3)_{sed}= NH3_{res}$
& 3 &
Ammonia NH3 [mg N/L]
\\

$Q(ON)_{sed} = (ON_{res} - ON_{sink})$
& 5 &
Organic Nitrogen ON [mg N/L]
\\

$Q(OPO4)_{sed}= OPO4_{res}$
& 6 &
\GBox{5}{
Inorganic Phosphorous \\
\HHSP OPO4 [mg P/L]
}
\\

$Q(OP)_{sed}= (OP_{res}-OP_{sink})$
& 7 &
Organic Phosphorous OP [mg P/L]
\\

\GBox{5}{
$Q(ON_{sed})= ON_{sink} - ON_{res} $
\HSP $\opn - NH3_{res}$
}
& 1sed &
\GBox{5}{
Sediment Organic Nitrogen \\
\HHSP ON${}_{sed}$ [mg N/L]
}
\\

\GBox{5}{
$Q(OP_{sed})= OP_{sink}- OP_{res}$
\HSP $\opn - OPO4_{res}$
}
& 2sed &
\GBox{5}{
Sediment Organic Phosphorous \\
\HHSP OP${}_{sed}$ [mg N/L]
}
\\


\hline
\end{tabular}
\caption{Sediment Mass balances}
\label{SMassBalance}
\end{table}




\begin{table}\centering
\begin{tabular}{lll}
\hline

\GBox{5}{
$NH3_{res}= Vol_{sed}*KNC_{sed}$
\HSP $\opn *KNT^{(T-T_0)}* ON_{sed}$
}
& 1 &
\GHBox{
Mineralization of organic nitrogen in sediment
}
\\

$ON_{res} = Vol_{sed}*KN_{res}*F_{vel}*ON_{sed}$
& 2 &
\GHBox{
Resuspention of organic nitrogen in sediment
}
\\

\GBox{5}{
$ON_{sink}= Vol*\frac{(1-exp(-dt/\tau_N))}{dt}$
\HSP $\opn *ON*FPON$
}
& 3 &
\GHBox{
Sink of organic nitrogen from the water column
}
\\

\GBox{5}{
$OPO4_{res}= Vol_{sed}*KPC_{sed}$
\HSP $\opn *KPT^{(T-T_0)}*OP_{sed}$
}
& 4 &
\GHBox{
Mineralization of organic phosphorous in sediment
}
\\

$OP_{res}= Vol_{sed} * KP_{res} * F_{vel} * OP_{sed}$
& 5 &
\GHBox{
Resuspention of organic phosphorous in sediment
}
\\

\GBox{5}{
$OP_{sink}= Vol*\frac{(1-exp(-dt/\tau_P))}{dt}$
\HSP $\opn *OP*FPOP$
}
& 6 &
\GHBox{
Sink of organic phosphorous from the water column
}
\\

$\tau_N = H/w_s$
& 7 &
\GHBox{
Time scale for sinking processes of organic N
}
\\

$\tau_P = H/w_s$
& 8 &
\GHBox{
Time scale for sinking processes of organic P
}
\\

\hline
\end{tabular}
\caption{Sediment functional expressions}
\label{SFuncDesc}
\end{table}







\begin{table}\centering
\begin{tabular}{lll}
\hline

$KNC_{sed} = 0.075$
&
Mineralization of sediment ON rate constant
\\

$KNT = 1.08$
&
Mineralization of sediment ON rate temperature constant
\\

$KPC_{sed} = 0.22$
&
Mineralization of sediment OP rate constant
\\

$KPT = 1.08$
&
Mineralization of sediment OP rate temperature constant
\\

$KN_{res} = 0.1$
&
Fraction of sediment depth resuspended/day
\\

$KP_{res} = 0.1$
&
Fraction of sediment depth resuspended/day
\\

$FPON = 0.5$
&
Fraction of particulate organic N
\\

$FPOP = 0.5$
&
Fraction of particulate organic P
\\

$F_{vel} = 1$
&
Velocity coefficient
\\

$w_s = 10$ m/day
&
Sinking velocity
\\

$T_0 = 20$ \Degree C
&
Optimal temperature value
\\

\hline
\end{tabular}
\caption{Sediment parameters}
\label{SParas}
\end{table}




\begin{table}\centering
\begin{tabular}{lll}
\hline

$Vol$
& [m\power{3}] &
volume
\\

$Vol_{sed}$
& [m\power{3}] &
sediment volume
\\

$H$
& [m] &
total depth of water column
\\

$dt$
& [sec] &
time step
\\

\hline
\end{tabular}
\caption{Sediment variables}
\label{SVars}
\end{table}







%\end{document}



%\end{document}
