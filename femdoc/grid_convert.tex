
Data files with boundary line and bathymetry should be given. These
files have to be transformed into |GRD| files, that can be read and
manipolated with the programs |mesh| and |grid|. Examples of how to do
so can be found in coast.pl and ldb.pl for the coastline and bathy.pl
for the bathymetry points.

\begin{code}
    coast.pl mpcoast.dat > coast.grd
    bathy.pl mpbathy.dat > depth.grd
\end{code}

Please note that the coordinates for the GRD files should be always in
meters. Therefore, if you have your coordinates in other units, you have
to adjust the conversion routines in order to create the new coordinates
in meters.

Please note that UTM coordinates are in meters, so UTM coordinates are
fine. However, since UTM coordinates are normally huge numbers, there
might be an accuracy problem when you try to create the grid. If this
happens, you should first shift your UTM coordinates so that the origin
of your new coordinate system coincides with the central point of your
grid. This translation can be done using the program |grd_transl.pl|.

Other transformation routines are:

\begin{itemize}

\item |dxf2grd.pl|  Transforms a grid from |DXF| (Autocad) to |GRD|
format. This is still experimental.

\item |kml2grd.pl|  Transforms a grid from the Google Earth format |KML|
to |GRD| format.

\item |xyz2grd.pl|  Transforms a simple list of nodes to |GRD|
format. Every line contains 3 values $(x,y,z)$ or two values $(x,y)$,
when the information on depth is missing.

\end{itemize}

Please note that for SHYFEM depth values have to be positive. If your
files have depth values as negative numbers, you will have to invert
them. You can use the command

\begin{code}
    grd_modify.pl -depth_invert grd-file
\end{code}

to achieve this task.

