\documentclass[12pt,draft]{article}

% 3d discretization

\usepackage{amssymb}
\usepackage{amsmath}

\newcommand{\beq}{\begin{equation}}
\newcommand{\eeq}{\end{equation}}
\newcommand{\beqm}{\begin{multline}}
\newcommand{\eeqm}{\end{multline}}

\newcommand{\Unod}{U}
\newcommand{\Vnod}{V}
\newcommand{\znod}{\zeta}
\newcommand{\Unew}{U^n}
\newcommand{\Vnew}{V^n}
\newcommand{\znew}{\zeta^n}
\newcommand{\Uold}{U}
\newcommand{\Vold}{V}
\newcommand{\zold}{\zeta}
\newcommand{\Udel}{\Delta U}
\newcommand{\Vdel}{\Delta V}
\newcommand{\zdel}{\Delta \zeta}

\newcommand{\UV}{\mathbf{U}}
\newcommand{\UVnew}{\mathbf{U}^n}
\newcommand{\UVdel}{\mathbf{\Delta U}}
\newcommand{\UVhat}{\hat{\mathbf{U}}}
\newcommand{\UVhatdel}{\mathbf{\Delta \hat{U}}}
\newcommand{\HH}{\mathbf{H}}
\newcommand{\Amat}{\mathbf{\bar{A}}}
\newcommand{\AmatI}{\mathbf{\bar{A}}^{-1}}
\newcommand{\FF}{\mathbf{F}}
\newcommand{\II}{\mathbf{I}}
\newcommand{\Ipart}{\mathbf{\partial I}}

\newcommand{\dt}{\Delta t}
\newcommand{\dz}{\Delta z}

\newcommand{\Ftilde}{\tilde{F}}
\newcommand{\Ftx}[1]{\tilde{F}^x_{#1}}
\newcommand{\Fty}[1]{\tilde{F}^y_{#1}}

\newcommand{\delx}{\partial_x}
\newcommand{\dely}{\partial_y}

\newcommand{\dpartt}[1]{\frac{\partial #1}{\partial t}}
\newcommand{\dpartx}[1]{\frac{\partial #1}{\partial x}}
\newcommand{\dparty}[1]{\frac{\partial #1}{\partial y}}
\newcommand{\dpartz}[1]{\frac{\partial #1}{\partial z}}
\newcommand{\dpartzz}[1]{\frac{\partial^2 #1}{\partial z^2}}

\newcounter{cms} % nel preambolo


\title{Treatment of the 3D equations in SHYFEM}
%\author{Georg Umgiesser}

\begin{document}

\maketitle

\section{Derivation of the discretized equations in time}

We start with the equations of momentum, already integrated over one model layer $l$:

\begin{equation}
	\dpartt{\Unod_l} - f\Vnod_l + gH_l\dpartx{\znod}
		= \Ftx{l} + \nu\dpartzz{\Unod_l}
\end{equation}
\begin{equation}
	\dpartt{\Vnod_l} + f\Unod_l + gH_l\dparty{\znod}
		= \Fty{l} + \nu\dpartzz{\Vnod_l}
\end{equation}
and the continuity equation
\begin{equation}
	\dpartt{\znod}
		+ \sum_l \dpartx{\Unod_l}
		+ \sum_l \dparty{\Vnod_l}
		= 0
\end{equation}


Here $\Unod_l$ and $\Vnod_l$ are the integrated velocities (transports) over layer
$l$ in direction $x$ and $y$,
$f$ the Coriolis parameter, $g$ the gravitational acceleration, $H_l$ the depth of
layer $l$, $\znod$ the water level,
$\nu$ the turbulent viscosity and $\Ftilde_l$ the other terms that will be
treated explicitly in the discretization below, like advection, horizontal diffusion,
atmospheric pressure etc., while $z$ denotes the vertical 
coordinate (positive pointing upward).

We discretize in time and develop semi-implicitly the Coriolis term ($\alpha_f$),
the barotropic pressure term ($\alpha_m$), the vertical stress term ($\alpha_d$)
and the divergence term in the continuity equation ($\alpha_z$).
A superscript ${}^n$ identifies the values at the new time step, 
whereas no superscript refers to the
old time step.

\begin{multline}
	\frac{\Unew_l-\Uold_l}{\dt} 
	- f(\alpha_f\Vnew_l + (1-\alpha_f)\Vold_l) 
	+ gH_l \delx(\alpha_m\znew + (1-\alpha_m)\zold) 
	= \\
	\Ftx{l} 
	+ \nu\alpha_d\frac{\Unew_{l-1}-2\Unew_l+\Unew_{l+1}}{\dz^2}
	+ \nu(1-\alpha_d)\frac{\Uold_{l-1}-2\Uold_l+\Uold_{l+1}}{\dz^2}
\end{multline}

\begin{multline}
	\frac{\Vnew_l-\Vold_l}{\dt} 
	+ f(\alpha_f\Unew_l + (1-\alpha_f)\Uold_l) 
	+ gH_l \dely(\alpha_m\znew + (1-\alpha_m)\zold)  
	= \\
	\Fty{l}
	+ \nu\alpha_d\frac{\Vnew_{l-1}-2\Vnew_l+\Vnew_{l+1}}{\dz^2}
	+ \nu(1-\alpha_d)\frac{\Vold_{l-1}-2\Vold_l+\Vold_{l+1}}{\dz^2}
\end{multline}

Here for simplicity a regular discretization in vertical layers
of depth $\dz$ has been assumed.

It is easy to see that with the definition
$\Udel = \Unew - \Uold$ we can rearrange
\beq
\alpha\Unew + (1-\alpha)\Uold = \alpha\Udel + \Uold
\eeq 
Using this definition and rearranging,
the two momentum equations become

\begin{multline}
	\frac{\Udel_l}{\dt} 
	- f\alpha_f\Vdel_l 
	- \nu\alpha_d\frac{\Udel_{l-1}-2\Udel_l+\Udel_{l+1}}{\dz^2}
	+ gH_l\alpha_m\delx\zdel 
	= \\
	\Ftx{l}
	+ f\Vold_l
	+ \nu\frac{\Uold_{l-1}-2\Uold_l+\Uold_{l+1}}{\dz^2}
	- gH_l\delx\zold 
\end{multline}

\begin{multline}
	\frac{\Vdel_l}{\dt} 
	+ f\alpha_f\Udel_l 
	- \nu\alpha_d\frac{\Vdel_{l-1}-2\Vdel_l+\Vdel_{l+1}}{\dz^2}
	+ gH_l\alpha_m\dely\zdel 
	= \\
	\Fty{l}
	- f\Uold_l
	+ \nu\frac{\Vold_{l-1}-2\Vold_l+\Vold_{l+1}}{\dz^2}
	- gH_l\dely\zold 
\end{multline}

In the equations above on the left hand side the unknowns 
that still depend on the values on the new time step
have been gathered,
whereas on the right hand side all the terms that are already known (explicit terms)
are listed. If we call the whole right hand side $F$ and use 
the following definitions
\begin{equation}
	\gamma=\dt f\alpha_f 
	\hspace{1cm}
	\epsilon=\dt\nu\alpha_d/\dz^2
	\hspace{1cm}
	\beta=\dt g\alpha_m
\end{equation}
we can write the discretized momentum equations as
\begin{multline}
	\Udel_l - \gamma\Vdel_l 
	- \epsilon(\Udel_{l-1}-2\Udel_l+\Udel_{l+1})
	+ \beta H_l\delx\zdel 
	= \dt F^x_l
\end{multline}
\begin{multline}
	\Vdel_l + \gamma\Udel_l 
	- \epsilon(\Vdel_{l-1}-2\Vdel_l+\Vdel_{l+1})
	+ \beta H_l\dely\zdel 
	= \dt F^y_l
\end{multline}

We now introduce a new notation, where we gather into one array
the single layer transports $\Unod_l$ and $\Vnod_l$
and level depths $H_l$: 
\begin{equation}
\UV = 
\begin{bmatrix}
 U_1 \\ V_1 \\ U_2 \\ V_2 \\ \vdots \\ U_L \\ V_L 
\end{bmatrix}
\hspace{1cm}
\UVdel =
\begin{bmatrix}
 \Udel_1 \\ \Vdel_1 \\ \Udel_2 \\ \Vdel_2 \\ \vdots \\
 \Udel_L \\ \Vdel_L 
\end{bmatrix}
\hspace{1cm}
\HH^x =
\begin{bmatrix}
 H_1 \\ 0 \\ H_2 \\ 0 \\ \vdots \\ H_L \\ 0 
\end{bmatrix}
\hspace{1cm}
\HH^y =
\begin{bmatrix}
 0 \\ H_1 \\ 0 \\ H_2 \\ \vdots \\ 0 \\ H_L 
\end{bmatrix}
\end{equation}
where $L$ denotes the total number of layers for this element.
With this notation we can collect the momentum equations into 
one array equation
\begin{equation}
	\label{momentum_intermediate}
	\Amat \UVdel + \beta(\HH^x\delx\zdel+\HH^y\dely\zdel)
	= \dt\FF 
\end{equation}
with $\FF$ is defined as
\begin{equation}
\FF = 
\begin{bmatrix}
 F^x_{1} & F^y_{1} & F^x_{2} & F^y_{2} & \cdots 
 & F^x_{L} & F^y_{L}  
\end{bmatrix}^T.
\end{equation}

Here $\Amat$ is a matrix that condenses all the information of
the implicit system into one matrix. Since one level is connected
to the level above and below and also within the level between 
$\Unod_l$ and $\Vnod_l$ this results in a pentagonal matrix
for one node. The structure of the matrix is given in the appendix.

After computing the inverse $\AmatI$ of $\Amat$ equation
(\ref{momentum_intermediate})
can be (at least formally) solved for $\UVdel$
\begin{equation}
	\UVdel =
		- \beta(\AmatI\HH^x\delx\zdel+\AmatI\HH^y\dely\zdel)
		+ \dt\AmatI\FF 
\end{equation}
In this equation the water level derivative is still unkown and
the term containing it
cannot yet be computed. However the last term is completely
known and can be already 
computed. We define a new auxiliary velocity as
\begin{equation}
	\label{uv_hat}
	\UVhat = \UV + \dt\AmatI\FF
\end{equation}
and therefore
$\UVhatdel = \dt\AmatI\FF$. With this definition we finally have
for the momentum equation
\begin{equation}
	\label{momentum_final}
	\UVdel = \UVhatdel
		- \beta 
		\left(
		\AmatI\HH^x\delx\zdel+\AmatI\HH^y\dely\zdel
		\right).
\end{equation}

In order to solve completely for the unknown velocities $\UVdel$
we will have to use the continuity equation. After discretization,
and taking into account that the divergence term is treated
semi-implicitly with weighting $\alpha_z$, the continuity
equation reads
\begin{equation}
	\frac{\zdel}{\dt}
	+ \sum_l \dpartx{}(\alpha_z\Unew_l+(1-\alpha_z)\Uold_l) 
	+ \sum_l \dparty{}(\alpha_z\Vnew_l+(1-\alpha_z)\Vold_l) 
	= 0
\end{equation}
and transforming to $\Udel$ and $\Vdel$ we have
\begin{equation}
	\zdel
	+ \dt \left[
	   \sum_l \dpartx{}(\alpha_z\Udel_l+\Uold_l) 
	+ \sum_l \dparty{}(\alpha_z\Vdel_l+\Vold_l) 
	\right] 
	= 0.
\end{equation}
We want to use the same vector notation as before. Therefore
we define the two arrays 
$\II^x$ and $\II^y$ and a combination of both
$\Ipart = \II^x\delx + \II^y\dely$ as
\begin{equation}
\II^x =
\begin{bmatrix}
1 \\ 0 \\ 1 \\ 0 \\ \vdots \\ 1 \\ 0 
\end{bmatrix}
\hspace{1cm}
\II^y =
\begin{bmatrix}
0 \\ 1 \\ 0 \\ 1 \\ \vdots \\ 0 \\ 1
\end{bmatrix}
\hspace{1cm}
\Ipart =
\begin{bmatrix}
\delx \\ \dely \\ \delx \\ \dely \\ \vdots \\ \delx \\ \dely
\end{bmatrix}
\end{equation}
and with this definition the continuity equation becomes
\begin{equation}
	\label{conti_final}
	\zdel
	+ \dt\alpha_z\Ipart\cdot\UVdel
	+ \dt\Ipart\cdot\UV
	= 0.
\end{equation}
Here the dot $\cdot$ indicates the scalar product. This transfers 
the formal scalar product in a summation.

We can now formally substitute in (\ref{conti_final})
$\UVdel$ from (\ref{momentum_final}). If this is done
we have
\begin{multline}
	\zdel
	- \dt\alpha_z\beta\; \Ipart\cdot
		(\AmatI\HH^x\delx+\AmatI\HH^y\dely)\zdel \\
	=
	- \dt[ \alpha_z\Ipart\cdot\UVhatdel
	+ \Ipart\cdot\UV ].
\end{multline}
We can rearrange this, using the definition of $\UVhatdel$ 
(\ref{uv_hat}) and
$\beta$ and 
defining a new $\delta=\dt\alpha_z\beta =
g\dt^2\alpha_z\alpha_m$. We arrive at
\begin{multline}
	[ 1
	- \delta\; \Ipart\cdot
		(\AmatI\HH^x\delx+\AmatI\HH^y\dely) ] \zdel \\
	=
	- \dt[ \alpha_z\Ipart\cdot\UVhat
	+ (1-\alpha_z)\Ipart\cdot\UV ].
\end{multline}
The right hand side is known ($\UVhat$ can be computed through
(\ref{uv_hat}) and the rest only involves horizotanl differentiation)
and can be computed readily. The left hand side involves the 
horizontal differentiation of $\zdel$. Therefore this leads to 
a matrix that involves all nodes of the grid. 

Once the solution of $\zdel$ is known over the whole domain,
equation (\ref{momentum_final}) can be used in a slightly
modified form to compute the velocity at the new time level
through the following equation:
\begin{equation}
	\label{uv_final}
	\UVnew = \UV
		- \beta 
		\left(
		\AmatI\HH^x\delx\zdel+\AmatI\HH^y\dely\zdel
		\right).
\end{equation}

%%%%%%%%%%%%%%%%%%%%%%%
%%%%%%%%%%%%%%%%%%%%%%%
%%%%%%%%%%%%%%%%%%%%%%%

\appendix

%%%%%%%%%%%%%%%%%%%%%%%
%%%%%%%%%%%%%%%%%%%%%%%
%%%%%%%%%%%%%%%%%%%%%%%


\section{Structure of the vertical matrix}

\subsection{Regular vertical discretization}

The matrix $\Amat$ that describes the implicit part of the 
vertical system is a pentagonal band matrix. It can therefore 
be solved with standard methods. Here an
example of the structure of
the matrix with $L=4$ is given for reference. 
For space limitation we use
the abbreviation $\tilde{\epsilon}=1+2\epsilon$.

{
\newcommand{\EE}{-\epsilon}
\newcommand{\ET}{\tilde{\epsilon}}
\newcommand{\GG}{\gamma}
% \cdots \vdots \ddots
\begin{equation}
%\AA = 
\begin{bmatrix}
\ET  & -\GG & \EE & 0      & 0    & 0 & 0 & 0 \\
\GG & \ET   & 0    & \EE   & 0    & 0 & 0 & 0 \\
\EE  & 0      & \ET & -\GG & \EE & 0 & 0 & 0\\
0     & \EE   & \GG & \ET  & 0    & \EE   & 0 & 0 \\
0 & 0 & \EE  & 0      & \ET & -\GG & \EE & 0 \\
0 & 0 & 0     & \EE   & \GG & \ET  & 0    & \EE \\
0 & 0 & 0 & 0 & \EE  & 0      & \ET & -\GG \\
0 & 0 & 0 & 0 & 0     & \EE   & \GG & \ET      
\end{bmatrix}
\end{equation}
}

The above representation only shows the typical structure
of the matrix and holds only for a regular layer spacing
of $\dz$. If the the layers have different thickness the parameter
$\epsilon$ and $\tilde{\epsilon}$ depend on the layer number
and have to be written as $\epsilon_l$ etc. Moreover, boundary conditions on the surface and the bottom layer have not been 
taken into account.

\subsection{Vertical discretization with variable layer thickness}

In order to account for variable layer thickness and viscosity
and to allow boundary conditiosn to be handled properly we will have to start one more time from the vertical stress term
$\nu\dpartzz{\Unod}$. In case of variable viscosity we can write
this term for layer $l$ as
\begin{equation}
\dpartz{} \left( \nu\dpartz{U} \right)_l =
\frac{ \left( \nu\dpartz{U} \right)^\uparrow_l 
 - \left( \nu\dpartz{U} \right)^\downarrow_l
}{H_l}
\end{equation}
where the superscript ${}^\uparrow$ identifies the upper
interface and ${}^\downarrow$ the lower one. If we call $l-1$
the upper and $l$ the lower interface of layer $l$ (see Fig ??), we have
\begin{equation}
\dpartz{} \left( \nu\dpartz{U} \right)_l =
\frac{1}{H_l} \left[
	\left( \frac{\nu}{\dz} \right)_{l-1} (U_{l-1}-U_l)
 	- \left( \frac{\nu}{\dz} \right)_{l} (U_l-U_{l+1})
\right]
\end{equation}
where the terms $\frac{\nu}{\dz}$ are taken at the upper or
lower interface of layer $l$. The values of $\dz$ will be
computed as
\begin{equation}
\dz_{l-1} = \frac{H_{l-1}+H_l}{2}
	\hspace{1cm}
\dz_{l} = \frac{H_{l+1}+H_l}{2}.
\end{equation}

\newcommand{\EUT}[1]{\tilde{\epsilon}^\uparrow_#1}
\newcommand{\EDT}[1]{\tilde{\epsilon}^\downarrow_#1}
\newcommand{\EU}[1]{\epsilon^\uparrow_#1}
\newcommand{\ED}[1]{\epsilon^\downarrow_#1}
\newcommand{\EC}[1]{\epsilon_#1}

We can now define new $\epsilon$ values for layer $l$ as
\begin{equation}
\EUT{l} = \frac{1}{H_l}
		\left( \frac{\nu}{\dz} \right)_{l-1}
	\hspace{1cm}
\EDT{l}  = \frac{1}{H_l}
		\left( \frac{\nu}{\dz} \right)_{l}
\end{equation}
and with these values we can write
\begin{equation}
\dpartz{} \left( \nu\dpartz{U} \right)_l =
\EUT{l}(U_{l-1}-U_l) - \EDT{l}(U_l-U_{l+1}) =
\EUT{l}U_{l-1} - (\EUT{l}+ \EDT{l})U_l + \EDT{l}U_{l+1}
\end{equation}
This is the expression that has to be substituted for the vertical
stress term in $F^x_l$ (and a similar expression for $F^y_l$).

In order to use these definitions also in matrix $\Amat$ we
define
\begin{equation}
\EU{l} = \alpha_d\dt\EUT{l}
	\hspace{1cm}
\ED{l} = \alpha_d\dt\EDT{l}
	\hspace{1cm}
\EC{l} =  1 + \EU{l} + \ED{l}
\end{equation}


Now we can finally rewrite the matrix $\Amat$ as
{
\newcommand{\GG}{\gamma}
% \cdots \vdots \ddots
\begin{equation}
%\AA = 
\begin{bmatrix}
\EC{1}  & -\GG & \ED{1} & 0      & 0    & 0 & 0 & 0 \\
\GG & \EC{1}   & 0    & \ED{1}   & 0    & 0 & 0 & 0 \\
\EU{2}  & 0      & \EC{2} & -\GG & \ED{2} & 0 & 0 & 0\\
0     & \EU{2}   & \GG & \EC{2}  & 0    & \ED{2}   & 0 & 0 \\
0 & 0 & \EU{3}  & 0      & \EC{3} & -\GG & \ED{3} & 0 \\
0 & 0 & 0     & \EU{3}   & \GG & \EC{3}  & 0    & \ED{3} \\
0 & 0 & 0 & 0 & \EU{4}  & 0      & \EC{4} & -\GG \\
0 & 0 & 0 & 0 & 0     & \EU{4}   & \GG & \EC{4}      
\end{bmatrix}
\end{equation}
}

\subsection{Boundary conditions at the surface and at
the bottom}

Boundary conditions have to be integrated into this matrix. First
the values of $\epsilon$  at the surface $\EUT{1}$ and at the 
bottom $\EDT{L}$ have to be 0, because they are refering
to differences between levels where one of the two do not
exist:
\begin{equation}
	\EUT{1} = \EDT{L} = 0.
\end{equation}

For the surface layer a term due to the wind stress has to
be added to the right hand side of the momentum equation
\begin{equation}
	F^x_1 := F^x_1 + (1/\rho_0)\tau^s_x
	= F^x_1 + (\rho_a/\rho_0)c_D|w|w_x
\end{equation}
\begin{equation}
	F^y_1 := F^y_1 + (1/\rho_0)\tau^s_y
	= F^y_1 + (\rho_a/\rho_0)c_D|w|w_y
\end{equation}
where $\rho_0$ is the water density, $\rho_a$ the air density,
$c_D$ the wind drag coefficient, $w_x,w_y$ the wind velocity
in direction $x,y$ and $|w|$ its modulus.

For the bottom layer a semi-implicit approach can be applied. The two terms to be considered are 
\begin{equation}
	(1/\rho_0)\tau^b_x = (c_B/H_L^2)|U_L|U_L
	\hspace{1cm}
	(1/\rho_0)\tau^b_y = (c_B/H_L^2)|U_L|V_L
\end{equation}
where $|U_L|$ is the modulus of the bottom transport. With $R=(c_B/H_L^2)|U_L|$, we can write
\begin{equation}
	(1/\rho_0)\tau^b_x 
	= \alpha_r R\Unew_L + (1-\alpha_r)R\Uold_L
	= \alpha_r R\Udel_L + R\Uold_L
\end{equation}
\begin{equation}
	(1/\rho_0)\tau^b_y 
	= \alpha_r R\Vnew_L + (1-\alpha_r)R\Vold_L
	= \alpha_r R\Vdel_L + R\Vold_L
\end{equation}
The last (explicit) part of these equations has to be subtracted
from $F^x,F^y$:
\begin{equation}
		F^x_L := F^x_L + R\Uold_L
	\hspace{1cm}
		F^y_L := F^y_L + R\Vold_L
\end{equation}
whereas the implicit part has to be integrated into 
matrix $\Amat$
\begin{equation}
		\EC{L} := \EC{L} + \alpha_r\dt R.
\end{equation}
This completes the treatment of the vertical stress term
and the set up of the matrix $\Amat$.


\begin{figure}[!t]

\begin{minipage}[b]{.99\textwidth}
\unitlength=1mm\centering
\begin{picture}(50,49)

\put(0,10){\line(1,0){40}}
\put(0,20){\line(1,0){40}}
\put(0,30){\line(1,0){40}}
%\put(0,40){\line(1,0){40}}
\put(0,50){\line(1,0){40}}
\put(0,60){\line(1,0){40}}
\put(0,70){\line(1,0){40}}

\put(10,12){\makebox(20,5){layer L}}
\put(10,22){\makebox(20,5){layer L-1}}
\put(10,52){\makebox(20,5){layer 2}}
\put(10,62){\makebox(20,5){layer 1}}

\put(50,8){\makebox(10,5){interface L}}
\put(50,18){\makebox(10,5){interface L-1}}
\put(50,28){\makebox(10,5){interface L-2}}
\put(50,48){\makebox(10,5){interface 2}}
\put(50,58){\makebox(10,5){interface 1}}
\put(50,68){\makebox(10,5){interface 0}}

\put(10,5){\makebox(20,5){bottom}}
\put(10,70){\makebox(20,5){surface}}

\end{picture}
\caption{Vertical layer discretization with zeta layers. The surface
layer is the first layer and layers until the bottom to layer $L$.}
\label{fig:vdiscr}
\end{minipage}
\end{figure}















\end{document}

%\begin{bmatrix}
%0 & \cdots & 0 
%\\ \vdots & \ddots & \vdots 
%\\ 0 & \cdots & 0 
%\end{bmatrix}

