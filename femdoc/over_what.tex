
The finite element program \shy{} is a program package that can be
used to resolve the hydrodynamic equations in lagoons, seas, estuaries
and lakes. The program uses finite elements for the resolution of
the hydrodynamic equations. These finite elements, together with an
effective semi-implicit time resolution algorithm, makes this program
especially suitable for applications in areas with a complicated geometry
and bathymetry.

The program \shy{} resolves the depth integrated shallow water equations
and can use both a two- and a three-dimensional formulation, depending
on the user's needs.

Finite elements are well adapted to problems dealing with complex
bathymetric situations and geometries.  The finite element method has
an advantage over other methods (e.g., finite differences) because it
allows more flexibility with its subdivision of the system in triangles
varying in form and size.  This flexibility can be used also in situations
where it is not desired to have uniform resolution of the whole basin,
but where a focus in resolution is needed only in some parts of the area.

It is possible to simulate shallow water flats, i.e., tidal marshes
that in a tidal cycle may be covered with water during high tide and
then fall dry during ebb tide. This phenomenon is handled by the model
in a mass conserving way.

Finite element methods have been introduced into hydrodynamics since 1973
and have been extensively applied to shallow water equations by numerous
authors \cite{Grotkop73, Taylor75, Herrling77, Herrling78, Holz82}.

% FIXME - new references

The model presented here \cite{Umgies86, Umgies93} uses the mathematical
formulation of the semi-implicit algorithm that decouples the solution
of the water levels and velocity components from each other leading to
smaller systems to solve. Models of this type have been presented from
1971 on by many authors \cite{Kwizak71, Duwe82, Backhaus83}.

