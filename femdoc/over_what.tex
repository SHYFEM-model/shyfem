

The finite element program \shy{} is a program package that can be used
to resolve the hydrodynamic equations in lagoons, coastal seas,
estuaries and lakes. The program uses finite elements for the
resolution of the hydrodynamic equations. These finite elements,
together with an effective semi-implicit time resolution algorithm,
makes this program especially suitable for application to a complicated
geometry and bathymetry.

This version of the program \shy{} resolves the depth integrated
shallow water equations. It is therefore recommended for the
application of very shallow basins or well mixed estuaries. Storm surge
phenomena can be investigated also.  This two-dimensional version of
the program is not suited for the application to baroclinic driven
flows or large scale flows where the the Coriolis acceleration is
important.

Finite elements are superior to finite differences when dealing with
complex bathymetric situations and geometries. Finite differences are
limited to a regular outlay of their grids. This will be a problem if
only parts of a basin need high resolution.  The finite element method
has an advantage in this case allowing more flexibility with its
subdivision of the system in triangles varying in form and size.

This model is especially adapted to run in very shallow basins. It is
possible to simulate shallow water flats, i.e., tidal marshes that in a
tidal cycle may be covered with water during high tide and then fall
dry during ebb tide. This phenomenon is handled by the model in a mass
conserving way.

Finite element methods have been introduced into hydrodynamics since
1973 and have been extensively applied to shallow water equations by
numerous authors \cite{Grotkop73, Taylor75, Herrling77, Herrling78, Holz82}.

The model presented here \cite{Umgies86, Umgies93} uses the mathematical
formulation of the semi-implicit algorithm that decouples the solution
of the water levels and velocity components from each other leading to
smaller systems to solve. Models of this type have been presented from
1971 on by many authors \cite{Kwizak71, Duwe82, Backhaus83}.


