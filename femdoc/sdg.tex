
%------------------------------------------------------------------------
%
%    Copyright (C) 1985-2020  Georg Umgiesser
%
%    This file is part of SHYFEM.
%
%    SHYFEM is free software: you can redistribute it and/or modify
%    it under the terms of the GNU General Public License as published by
%    the Free Software Foundation, either version 3 of the License, or
%    (at your option) any later version.
%
%    SHYFEM is distributed in the hope that it will be useful,
%    but WITHOUT ANY WARRANTY; without even the implied warranty of
%    MERCHANTABILITY or FITNESS FOR A PARTICULAR PURPOSE. See the
%    GNU General Public License for more details.
%
%    You should have received a copy of the GNU General Public License
%    along with SHYFEM. Please see the file COPYING in the main directory.
%    If not, see <http://www.gnu.org/licenses/>.
%
%    Contributions to this file can be found below in the revision log.
%
%------------------------------------------------------------------------

%
% $Id: discret.tex,v 1.13 2004/03/29 13:46:19 georg Exp georg $
%
% todo:
%
%%%%%%%%%%%%%%%%%%%%%%%%%%%%%%%%%%%%%%%%%%%%%%%%%%%%%%%%%%
%%%%%%% basic heading for Latex %%%%%%%%%%%%%%%%%%%%%%%%%%
%%%%%%%%%%%%%%%%%%%%%%%%%%%%%%%%%%%%%%%%%%%%%%%%%%%%%%%%%%

\documentclass[a4paper]{report}
%\documentclass[a4paper]{book}
%\documentclass[a4paper,draft]{book}

%\usepackage{natbib}
\usepackage{graphicx}

%\usepackage{showkeys,layout}
%\usepackage{amssymb}
\usepackage{ifthen}

\usepackage{fancyvrb}
%\usepackage{shortvrb}
%\usepackage{alltt}              % as verbatim, but interpret \ { }

\newcommand{\shy}{{\tt SHYFEM}}
\newcommand{\FFF}{{\tt FORTRAN}}
%\newcommand{\VERSION}{4.31}

\newcommand{\descrpsep}{\vspace{0.2cm}}
\newcommand{\descrpitem}[1]{\descrpsep\parbox[t]{2cm}{#1}}
\newcommand{\descrptext}[1]{\parbox[t]{10cm}{#1}\descrpsep}
\newcommand{\densityunit}{kg\,m${}^{-3}$}
\newcommand{\accelunit}{m\,s${}^{-2}$}
\newcommand{\maccelunit}{m${}^{4}$\,s${}^{-2}$}
\newcommand{\dischargeunit}{m${}^3$\,s${}^{-1}$}

%%%%%%%%%%%%%%%%%%%%%%%%%%%%%%%%%%%%%%%%%%%%%%%%%%%%%%%%%%
%%%%%%% user commands %%%%%%%%%%%%%%%%%%%%%%%%%%%%%%%%%%%%
%%%%%%%%%%%%%%%%%%%%%%%%%%%%%%%%%%%%%%%%%%%%%%%%%%%%%%%%%%


\newcommand{\filedir}{../eps/}		%file directory
\newcommand{\fileprefix}{}              %file prefix

\newcommand{\filesize}{}
%\newcommand{\filesize}{_small}

\newcommand{\fileid}{\fileprefix\filesize}

%%%%%%%%%%%%%%%%%%%%%%%%%%%%%%%%%%%%%%%%%%%%%%%%%%%%%%%%%%

%\input{figs}
%\input{formulas}

\newcommand{\powm}[1]{${}^{#1}$}
\newcommand{\subm}[1]{${}_{#1}$}
\newcommand{\subrm}[1]{_{\mathrm #1}}
\newcommand{\timesm}{$\times$}
\newcommand{\pmm}{$\pm$}
\newcommand{\degm}{${}^o$}
\newcommand{\missing}{{\bf MISSING}}

\newcommand{\Celsius} {${}^{o}$C}
\newcommand{\power} [1] {${}^{#1}$}
\newcommand{\mod} [1] {|#1|}
\newcommand{\text} [1] {\textrm{#1}}
\newcommand{\ttggu} [1] {\texttt{#1}}
\newcommand{\ital} [1] {{\it #1}}
\newcommand{\ksunits} {m${}^{1/3}$s${}^{-1}$}

\newcommand{\beq} {\begin{equation}}
\newcommand{\eeq} {\end{equation}}
\newcommand{\beqa} {\begin{eqnarray}}
\newcommand{\eeqa} {\end{eqnarray}}
\newcommand{\beqas} {\begin{eqnarray*}}
\newcommand{\eeqas} {\end{eqnarray*}}

%----------------------------------------------------
\DefineVerbatimEnvironment%
{code}{Verbatim}
{numbers=left,numbersep=5pt,
frame=lines,framerule=0.8mm}
\newcommand{\str} [1] {\VerbatimInput{#1}}
%----------------------------------------------------

\newcommand{\eqref} [1] {\ref{eq:#1}}
\newcommand{\tabref} [1] {\ref{tab:#1\fileprefix}}
\newcommand{\figref} [1] {\ref{fig:#1\fileprefix}}
\newcommand{\sectref} [1] {\ref{sect:#1}}
%\newcommand{\pageref} [1] {\ref{page:#1}}

\newcommand{\bibref} [1] {\citep{#1}}   %(name, year)
\newcommand{\bibnref} [1] {\cite{#1}}   %name (year)

\newcommand{\Tab} {Tab.~}
\newcommand{\Fig} {Fig.~}
\newcommand{\Figs} {Figs.~}
\newcommand{\Tabs} {Tabs.~}
\newcommand{\tab} {Tab.~}
\newcommand{\fig} {Fig.~}
\newcommand{\figs} {Figs.~}
\newcommand{\tabs} {Tabs.~}

%%%%%%%%%%%%%%%%%%%%%%%%%%%%%%%%%%%%%%%%%%%%%%%%%%%%%%%%%%

\newcommand{\singlespacing} {\def\baselinestretch{1}}
\newcommand{\doublespacing} {\def\baselinestretch{1.7}}

\newcommand{\PTF}{15\%}		%percentage of tidal flats
\newcommand{\PFF}{18\%}		%percentage of fishing farms

\newcommand{\stacco}[1]{\vspace{0.5cm}{\noindent\bf #1}\vspace{0.5cm}}

\hyphenation{ ca-li-bra-tion }
\hyphenation{ eli-mi-na-tion eli-mi-na-ted }

\parindent 0cm

\DefineShortVerb{\|}

%%%%%%%%%%%%%%%%%%%%%%%%%%%%%%%%%%%%%%%%%%%%%%%%%%%%%%%%%%
%%%%%%% end of user commands %%%%%%%%%%%%%%%%%%%%%%%%%%%%%
%%%%%%%%%%%%%%%%%%%%%%%%%%%%%%%%%%%%%%%%%%%%%%%%%%%%%%%%%%

\title{%
	Software Development Guide for SHYFEM
}

\author{%
Georg Umgiesser
\\Istituto di Scienze Marine, ISMAR-CNR,
\\Castello 1364/A, 30122 Venezia, Italy
}

%%%%%%%%%%%%%%%%%%%%%%%%%%%%%%%%%%%%%%%%%%%%%%%%%%%%%%%%%%%%%%%%
%%%%%%%%%%%%%%%%%%%%%%%%%%%%%%%%%%%%%%%%%%%%%%%%%%%%%%%%%%%%%%%%
% Start of document
%%%%%%%%%%%%%%%%%%%%%%%%%%%%%%%%%%%%%%%%%%%%%%%%%%%%%%%%%%%%%%%%
%%%%%%%%%%%%%%%%%%%%%%%%%%%%%%%%%%%%%%%%%%%%%%%%%%%%%%%%%%%%%%%%

\begin{document}

%%%%%%%%%%%%%%%%%%%%%%%%%%%%%%%%%%%%%%%%%%%%%%%%%%%%%%%%%%%%%%%%
%%%%%%%%%%%%%%%%%%%%%%%%%%%%%%%%%%%%%%%%%%%%%%%%%%%%%%%%%%%%%%%%
% Main chapters
%%%%%%%%%%%%%%%%%%%%%%%%%%%%%%%%%%%%%%%%%%%%%%%%%%%%%%%%%%%%%%%%
%%%%%%%%%%%%%%%%%%%%%%%%%%%%%%%%%%%%%%%%%%%%%%%%%%%%%%%%%%%%%%%%

\chapter{Introduction}

This document describes some techniques for developing
software for the SHYFEM program. However, many, if not most of the
guidelines given in this report can be applied to general 
software development in FORTRAN. 

It describes how you should write clear programs that other people will
be able to read, debug, correct and extend. 

\chapter{General Rules}

This chapter gives general guidelines how to develop programs in
FORTRAN. Often you will find that the guidelines will be applicable 
to other programming languages.


\section{Overall rules}

\subsection{Always use explicit {\tt implicit} type declaration}

Even if \FFF{} can do without declaration of variables, please always
use the |implicit none| specification statement. It might be tempting to
leave out this statement and save some seconds in typing all the vriables that
you have used in your routine, but this is a terrible error. The seconds
you save from typing the declarations will be certainly be outweight by
the hours you will spend in debugging programs with subtil errors due
to non declaring your variables.

See for example this code fragment:

\begin{code}
	integer icount
	...
	if( a .eq. b ) then icount = icount + 1
\end{code}

If you have enabled type checking with |implicit none| your compiler
will signal you immediately the problem. But if not it might take
you hours to find out why |icount| is not incremented properly.

If you have type check enabled, the compiler will tell you that you are
using a variable |thenicount| which has not been declared, and you
will be able to eliminate the error from the beginning and correct the 
statement to

\begin{code}
	if( a .eq. b ) icount = icount + 1
\end{code}

Otherwise, you might have a nice debugging session.


\str{basic.str}

%%%%%%%%%%%%%%%%%%%%%%%%%%%%%%%%%%%%%%%%%%%%%%%%%%%%%%%%%%%%%%%%

\end{document}



comment your code
always use the same delimiters
do not use obsolete and deprecated features
indent your code consistently
how to read a file with unknown length
dimension check
assertions

how to inset new features

	create a new file or files
	limit the insertion point to the absolute minimum

refactor

	if more changes are needed
		-> first implement the changes
		-> check if all works
		-> then implement the new feature






