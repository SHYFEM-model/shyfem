
In summary, the following steps have to be carried out before you will
be able to run the model:

\begin{itemize}

\item Get the distribution and unpack it in a place of your choice.

\item Move to the root of the distribution (\ttt{cd \shydir}).

\item Check if all software is available (|make check_software|). This
step has to be done only the first time you install SHYFEM on a computer.

\item Install the model (|make install|). This has to be done everytime
you get and install a new version of the model.

\item Adjust options in |Rules.make|. This has to be done everytime you
change options (compiler, parallel execution, etc.). After this you have
to run also |make cleanall|.

\item Adjust dimension parameters in |Rules.make|. This has to be done the
first time and every time you change application (basin, etc.) to adapt
the dimensions to the new problem. You might also run |make cleanall|
after this step, but it is not required.

\item Compile the programs with |make fem| and have a look at the error
messages.

\end{itemize}


